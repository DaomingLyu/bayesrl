%##############################################################################
% Preamble
%##############################################################################

\documentclass{pset}
\name{Dustin Tran, Xiaomin Wang, Rodrigo Gomes}
\email{\{trandv,xiaominw, rgomes\}@mit.edu}

\course{6.834J/16.412J, S15}
\instructor{Professor Brian Williams}
\assignment{Problem Set \#3}
\duedate{April 17, 2015}

\begin{document}

%##############################################################################
% Begin Document
%##############################################################################

We follow the directory structure specified in the problem set, with two
exceptions:
\begin{itemize}
\item \texttt{documentation/} does not exist. Instead, documentation is written
in the \texttt{README.md} inside the current working directory. Any additional
documentation not purely necessary for the problem set submission is in the
Github wiki.
\item \texttt{source/} is named \texttt{bayesrl/} in order to follow Python
convention for installing modules.
\end{itemize}

%##############################################################################
% End Document
%##############################################################################

\end{document}
