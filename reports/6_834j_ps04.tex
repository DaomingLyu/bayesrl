%##############################################################################
% Preamble
%##############################################################################

\documentclass{pset}
\name{Dustin Tran, Xiaomin Wang, Rodrigo Gomes}
\email{\{trandv,xiaominw, rgomes\}@mit.edu}

\course{6.834J/16.412J, S15}
\instructor{Professor Brian Williams}
\assignment{Problem Set \#4}
\duedate{May 13, 2015}

\begin{document}

%##############################################################################
% Begin Document
%##############################################################################

\begin{center}
\Large Task: Robot grocery shopping in partially observable settings
\end{center}
\section{Motivation}
\label{sec:motivation}
Imagine that you're in your bed, hungry, and you just don't want to walk
all the way to the grocery store. Equivalently, imagine you're working, doing the
most exciting task to save humanity, and you have no time to grab food---menial
tasks are beyond you. We'd like a robot which can intelligently learn to shop
groceries for us: it understands a query from the user, moves to the
grocery store, finds the relevant (and cheapest) items, purchases it, and comes
back to the user.

In our project we focus on the arguably most difficult of
these tasks, which is to locate the groceries in the store. Moreover, we work
under the realistic
scenario in which the robot can only observe its surroundings (POMDP) rather
than an understanding of where it precisely is in the store (MDP).

Given this partially observable setting, the robot should learn how to obtain
all items in the grocery store and do so in an optimal amount of time. It must
learn: 1. how to navigate around the store without bumping into walls or aisles;
2. intelligently search for the items by learning which aisle corresponds to
which category; and 3. find the optimal path and sequence to obtain all items.

\section{Procedure}
\label{sec:procedure}

\section{Experiments}
\label{sec:experiments}

\bibliography{6_834j_ps04}
\bibliographystyle{plain}

%##############################################################################
% End Document
%##############################################################################

\end{document}
